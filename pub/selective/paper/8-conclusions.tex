\section{Conclusions}\label{sec-conclusions}

We have introduced selective functors, an abstraction between applicative
functors and monads. Like applicative functors, selective functors require all
effects to be known statically, before the execution starts. Like monads,
selective functors allow for effects to depend on values of earlier effects but
in a limited way: it is possible to skip some of the effects, but not create
new ones. In this sense selective functors allow you to describe computations
that are very much like hardware circuits: statically fixed, yet dynamically
reconfigurable.

We have demonstrated usefulness of the new abstraction on several examples, and
hope that the reader will find it useful in their next project too.
