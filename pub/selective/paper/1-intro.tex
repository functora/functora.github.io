\vspace{-0.5mm}
\section{Introduction}\label{sec-intro}

\emph{Monads}, introduced to functional programming
by~\citet{1995_wadler_monads}, are a powerful and general approach for
describing effectful (or impure) computations using pure functions. The key
ingredient of the monad abstraction is the \emph{bind} operator, denoted by
\hs{>>=} in Haskell\footnote{We use Haskell throughout this paper, but the
presented ideas are not language specific. We release two libraries for
selective applicative functors along with this paper, written in Haskell
(\cmd{https://hackage.haskell.org/package/selective})
and~OCaml~(\cmd{https://opam.ocaml.org/packages/selective}). The ideas have also
been translated to Coq~\citep{selective2019coq},
Kotlin~\citep{selective2019kotlin}, PureScript~\citep{selective2018purescript},
Scala~\citep{selective2019scala} and Swift~\citep{selective2019swift}.}:

\vspace{1mm}
\begin{minted}[xleftmargin=10pt]{haskell}
(>>=) :: Monad f => f a -> (a -> f b) -> f b
\end{minted}
\vspace{1mm}

\noindent
The operator takes two arguments: an effectful computation \hs{f}~\hs{a}, which
yields a value of type~\hs{a} when executed, and a recipe, i.e. a pure function
of type \hs{a}~\hs{->}~\hs{f}~\hs{b}, for turning~\hs{a} into a subsequent
computation of type \hs{f}~\hs{b}. This approach to composing effectful
computations is inherently sequential: until we execute the effects in
\hs{f}~\hs{a}, there is no way of obtaining the computation \hs{f}~\hs{b}, i.e.
these computations must be performed in sequence. The ability to enforce
a \emph{sequential execution} order is crucial for non-commutative effects, such
as printing to the terminal. Furthermore, the dependence between subsequent
effects can be used for \emph{conditional effect execution}, as demonstrated
below.

Consider a simple example, where we use the monad \hs{f}~\hs{=}~\hs{IO} to
describe an effectful program that prints \hs{"pong"} to the terminal if the
user enters \hs{"ping"}:

\vspace{1mm}
\begin{minted}[xleftmargin=10pt]{haskell}
pingPongM :: IO ()
pingPongM = getLine >>= \s -> if s@\,@==@\,@"ping" then putStrLn "pong" else pure ()
\end{minted}
\vspace{1mm}

\noindent
The first argument of the bind operator reads a string using
\hs{getLine}~\hs{::}~\hs{IO}~\hs{String}, and the second argument is the
function of type \hs{String}~\hs{->}~\hs{IO}~\hs{()}, which prints \hs{"pong"}
when~\hs{s}~\hs{==}~\hs{"ping"}.

As we will see in sections~\S\ref{sec-static} and~\S\ref{sec-haxl}, in some
applications it is desirable to know all possible effects \emph{statically},
i.e. \emph{before the execution}. Alas, this is not possible with monadic effect
composition. To \emph{inspect} the function \hs{\s}~\hs{->}~\hs{...}, we need
a string~\hs{s}, which becomes available only \emph{during execution}. We are
therefore unable to predict the effects that \hs{pingPongM} might perform:
instead of conditionally executing \hs{putStrLn}, as intended, it might delete a
file from disk, or launch proverbial missiles.

\emph{Applicative functors}, introduced by~\citet{mcbride2008applicative}, can
be used for composing statically known collections of effectful computations, as
long as these computations are \emph{independent} from each other. The key
ingredient of applicative functors is the \emph{apply} operator, denoted
by~\hs{<*>}:

\vspace{1mm}
\begin{minted}[xleftmargin=10pt]{haskell}
(<*>) :: Applicative f => f (a -> b) -> f a -> f b
\end{minted}
\vspace{1mm}

\noindent
The operator takes two effectful computations, which --- independently ---
compute values of types \hs{a}~\hs{->}~\hs{b} and \hs{a}, and returns their
composition that performs both computations, and then applies the obtained
function to the obtained value producing the result of type \hs{b}. Crucially,
both arguments and associated effects are known statically, which, for example,
allows us to pre-allocate all necessary computation resources upfront
(\S\ref{sec-static}) and execute all computations in parallel
(\S\ref{sec-haxl}).

Our ping-pong example cannot be expressed using applicative functors. Since the
two computations must be independent, the best we can do is to print \hs{"pong"}
unconditionally:

\vspace{0.5mm}
\begin{minted}[xleftmargin=10pt]{haskell}
pingPongA :: IO ()
pingPongA = fmap (\s -> id) getLine <*> putStrLn "pong"
\end{minted}
\vspace{0.5mm}

\noindent
Here we use \hs{fmap}~\hs{(\s}~\hs{->}~\hs{id)} to replace the input string
\hs{s}, which we now have no need for, with the identity function
\hs{id}~\hs{::}~\hs{()}~\hs{->}~\hs{()}, thus matching the type of
\hs{putStrLn}~\hs{"pong"}~\hs{::}~\hs{IO}~\hs{()}. We cannot execute the
\hs{putStrLn}~\hs{"pong"} effect conditionally but, on the positive side, the
effects are no longer hidden behind opaque effect-generating functions, which
makes it possible for the applicative functor \hs{f}~\hs{=}~\hs{IO} to
statically know the two effects embedded in \hs{pingPongA}.

At this point the reader is hopefully wondering: can we combine the advantages
of applicative functors and monads, i.e. allow for conditional execution of some
effects while retaining the ability to statically know all effects embedded in
a computation? It will hardly be a surprise that the answer is positive, but it
is far from obvious what the right abstraction should be. For example, one might
consider adding a new primitive called \hs{whenS} to \hs{IO}:

\vspace{1mm}
\begin{minted}[xleftmargin=10pt]{haskell}
whenS :: IO Bool -> IO () -> IO ()
\end{minted}
\vspace{1mm}

\noindent
This primitive executes the first computation, and then uses the obtained
\hs{Bool} to decide whether to execute the second computation or skip it. Let us
rewrite the ping-pong example using \hs{whenS}:

\vspace{1mm}
\begin{minted}[xleftmargin=10pt]{haskell}
pingPongS :: IO ()
pingPongS = whenS (fmap (=="ping") getLine) (putStrLn "pong")
\end{minted}
\vspace{1mm}

\noindent
We replace the input string~\hs{s} with \hs{True} if it is equal to \hs{"ping"},
and \hs{False} otherwise, thereby appropriately \emph{selecting} the subsequent
effectful computation. This approach gives us both conditional execution of
\hs{putStrLn}~\hs{"pong"}, and static visibility of both effects
(see~\S\ref{sec-free-ping-pong}). Crucially, \hs{whenS} must be an \hs{IO}
primitive instead of being implemented in terms of the monadic bind (\hs{>>=}),
because the latter would result in wrapping \hs{putStrLn}~\hs{"pong"} into an
opaque function, as in \hs{pingPongM}.

The main idea of this paper is that \hs{whenS}, as well as many other similar
combinators, can be seen as special cases of a new intermediate abstraction,
called \emph{selective applicative functors}, whose main operator for composing
effectful computations is \emph{select}:

\vspace{1mm}
\begin{minted}[xleftmargin=10pt]{haskell}
select :: Selective f => f (Either a b) -> f (a -> b) -> f b
\end{minted}
\vspace{1mm}

\noindent
Intuitively, the first effectful computation is used to select what happens
next: if it yields a \hs{Left}~\hs{a} you \emph{must execute} the second
computation in order to produce a \hs{b} in the end; otherwise, if it yields a
\hs{Right}~\hs{b}, you \emph{may skip} the subsequent effect, because you have
no use for the resulting function. Note the possibility of \emph{speculative
execution}: in some contexts, we can execute both computations in parallel,
cancelling the second computation if/when the first one evaluates to a
\hs{Right}~\hs{b}.

\vspace{1mm}
The contributions of this paper are as follows:

% \vspace{-1mm}
\begin{itemize}
    \item We introduce \emph{selective applicative functors} as a general
    abstraction situated between applicative functors and monads, characterising
    the relationships between all three abstractions with a set of laws, and
    defining a few important instances (\S\ref{sec-selective}).
    \item We discuss applications of the abstraction on two industrial case
    studies: the OCaml build system \Dune~\citep{dune} (\S\ref{sec-static}) and
    Facebook's \Haxl library~\cite{marlow2014haxl} (\S\ref{sec-haxl}).
    \item We present \emph{free selective applicative functors} and show how to
    use them to implement embedded domain-specific languages with both
    conditional effects and static analysis (\S\ref{sec-free}).
\end{itemize}

\noindent
We discuss alternatives to selective applicative functors and related work in
sections~\S\ref{sec-alternatives} and \S\ref{sec-related}.
